\documentclass[11pt,a4paper]{article}

\usepackage{etex} %évite les erreurs no room for ....

\usepackage[french]{babel}
\usepackage[utf8]{inputenc}
\usepackage[T1]{fontenc}
\usepackage{hyperref}
\usepackage{amsmath}
\usepackage{amssymb} %gestion des symboles mathmatiques
\usepackage{amsfonts} %gestion des polices mathmatiques
\usepackage{fancybox} %gestion des encadrements
\usepackage{lastpage} %gestion du nombre total de pages du document
\usepackage{geometry} %gestion des marges du document
\usepackage{fancyhdr} %gestion des entetes et des pieds de page
\usepackage{epsfig}
\usepackage[dvipsnames]{xcolor}
\usepackage{pdftricks}
\usepackage{eurosym}
\usepackage{pifont}
\usepackage{eucal}
\usepackage{multirow}
\usepackage{tabularx}
\usepackage{ulem}
\usepackage{dcolumn}
\usepackage{textcomp}
\usepackage{diagbox}
\usepackage{lscape}
\usepackage{ifpdf}
\usepackage[tikz]{bclogo}
\usepackage{amsthm}
\usepackage{tabvar}
\usepackage{multicol}
\usepackage{frcursive}
\usepackage{color}
\usepackage{colortbl}
\usepackage{pgf,tikz}
\usetikzlibrary{arrows}
\usepackage{eurosym}
\usepackage{cancel}
\usepackage{longtable,booktabs}
\usepackage{tcolorbox}
\tcbuselibrary{skins}
\tcbuselibrary{breakable}
\usepackage{varwidth}
\usepackage{MnSymbol} % contient le caractère \nsubset
\usepackage{wrapfig}
\usepackage{wasysym} % pour certains symboles
\usepackage{minted}
\usepackage{titlesec}
\usepackage[np]{numprint}
\usepackage{fontawesome5}
\usepackage{xspace}

\titleformat{\section}
{\Large\bfseries}{\thesection}{1em}{}[\titlerule]
% {\Large\bfseries}{\thesection}{1em}{\MakeUppercase}[\titlerule]

\titleformat{\subsection}
{\large\bfseries}{\thesubsection}{1em}{}[\titlerule]
% {\large\bfseries}{\thesubsection}{1em}{\MakeUppercase}[\titlerule]

%% Macro pour limite  gauche%%
\newcommand{\limgauche}[2]{\lim_{#1\rightarrow #2\hspace{-1.5em}\raisebox{-1.2ex}{\scriptsize $#1<#2$}}}

%% Macro pour limite  droite%%
\newcommand{\limdroite}[2]{\lim_{#1\rightarrow #2\hspace{-1.5em}\raisebox{-1.2ex}{\scriptsize $#1>#2$}}}

\newcommand{\R}{\mathbb{R}}
\newcommand{\N}{\mathbb{N}}
%\newcommand{\D}{\mathbb{D}}
\newcommand{\Z}{\mathbb{Z}}
\newcommand{\Q}{\mathbb{Q}}
\newcommand{\C}{\mathbb{C}}
\newcommand{\lm}[2]{\displaystyle{\lim_{#1 \rightarrow #2}}}
\newcommand{\e}[1]{\text{e}^{#1}}
\renewcommand{\theenumi}{\textbf{\arabic{enumi}}}
\renewcommand{\labelenumi}{\textbf{\theenumi.}}
\renewcommand{\theenumii}{\textbf{\alph{enumii}}}
\renewcommand{\labelenumii}{\textbf{\theenumii.}}
\newcommand{\vect}[1]{\mathchoice%
{\overrightarrow{\displaystyle\mathstrut#1\,\,}}%
{\overrightarrow{\textstyle\mathstrut#1\,\,}}%
{\overrightarrow{\scriptstyle\mathstrut#1\,\,}}%
{\overrightarrow{\scriptscriptstyle\mathstrut#1\,\,}}}
\def\Oij{$\left(\text{O},~\vect{\imath},~\vect{\jmath}\right)$}
\def\Oijk{$\left(\text{O},~\vect{\imath},~ \vect{\jmath},~ \vect{k}\right)$}
\def\Ouv{$\left(\text{O},~\vect{u},~\vect{v}\right)$}
\newcommand{\V}{\overrightarrow}
\newcommand{\Coor}{\binom}
\newcommand{\Rep}{\left(O;\V{i},\V{j}\right)}
\renewcommand{\emph}{\textit}
\setlength{\emergencystretch}{3em} % prevent overfull lines
\providecommand{\tightlist}{\setlength{\itemsep}{0pt}\setlength{\parskip}{0pt}}

\newcommand{\Syst}[2]{\left\{
  \begin{array}{ccccc}
    #1   
    #2
  \end{array}\right.
}
\usepackage{listingsutf8}
%configuration de listings
\definecolor{couleuroperations}{rgb}{0.6,0.2,1}
\lstset{
  backgroundcolor=\color{lightgray!15},
  frame=leftline,
%   framexleftmargin=5mm,
  rulesepcolor=\color{lightgray},
inputencoding=utf8,
language=python,
basicstyle=\ttfamily\small, %
identifierstyle=\color{black}, %
keywordstyle=\color{MidnightBlue}, %
stringstyle=\color{purple}, %
commentstyle=\it\color{olive}, %
columns=flexible, %
tabsize=2, %
extendedchars=true, %
showspaces=false, %
showstringspaces=false, %
numbers=none, %
numberstyle=\tiny, %
breaklines=true, %
breakautoindent=true, %
captionpos=b,
literate=%
    *{+}{{{\color{couleuroperations}+}}}1
    {-}{{{\color{couleuroperations}-}}}1
    {/}{{{\color{couleuroperations}/}}}1
    {*}{{{\color{couleuroperations}*}}}1
    {=}{{{\color{couleuroperations}=}}}1
    {<}{{{\color{couleuroperations}<}}}1
    {>}{{{\color{couleuroperations}>}}}1
    {\%}{{{\color{couleuroperations}\%}}}1
    {(}{{{\color{couleuroperations}(}}}1
    {)}{{{\color{couleuroperations})}}}1
    {[}{{{\color{couleuroperations}[}}}1
    {]}{{{\color{couleuroperations}]}}}1
    {.}{{{\color{couleuroperations}.}}}1
    {,}{{{\color{couleuroperations},}}}1
    {1}{{{\color{Cyan}1}}}1
    {2}{{{\color{Cyan}2}}}1
    {3}{{{\color{Cyan}3}}}1
    {4}{{{\color{Cyan}4}}}1
    {5}{{{\color{Cyan}5}}}1
    {6}{{{\color{Cyan}6}}}1
    {7}{{{\color{Cyan}7}}}1
    {8}{{{\color{Cyan}8}}}1
    {9}{{{\color{Cyan}9}}}1
    {0}{{{\color{Cyan}0}}}1,
}

%%%%%%%%%%%%%%%%%% MES ENVIRONNEMENTS %%%%%%%%%%%%%%%%%%%%%%%%%%%%%

\newenvironment{definition}[1]{\begin{tcolorbox}[title= {\color{NavyBlue} \faPencil*}~~\textbf{Définition #1}, colframe=NavyBlue, colback=white, colbacktitle=NavyBlue!10!white, coltitle=black, boxrule=0.1mm, titlerule=0mm]}{\end{tcolorbox}}

\newenvironment{info}[1]{\begin{tcolorbox}[title= {\color{ProcessBlue} \faBook}~~\textbf{#1}, colframe=ProcessBlue, colback=white, colbacktitle=ProcessBlue!15!white, coltitle=black, boxrule=0.1mm, titlerule=0mm]}{\end{tcolorbox}}

\newenvironment{remarque}[1]{\begin{tcolorbox}[title= {\color{SkyBlue} \faInfo}~~\textbf{Remarque #1}, colframe=SkyBlue, colback=white, colbacktitle=SkyBlue!20!white, coltitle=black, boxrule=0.1mm, titlerule=0mm]}{\end{tcolorbox}}

\newenvironment{indication}[1]{\begin{tcolorbox}[title= {\color{Turquoise} \faHotjar}~~\textbf{Indication #1}, colframe=Turquoise, colback=white, colbacktitle=Turquoise!15!white, coltitle=black, boxrule=0.1mm, titlerule=0mm]}{\end{tcolorbox}}

\newenvironment{solution}[1]{\begin{tcolorbox}[title= {\color{ForestGreen} \faCheck}~~\textbf{Solution #1}, colframe=ForestGreen, colback=white, colbacktitle=ForestGreen!15!white, coltitle=black, boxrule=0.1mm, titlerule=0mm]}{\end{tcolorbox}}

\newenvironment{exercice}[1]{\begin{tcolorbox}[title= {\color{YellowGreen} \faQuestion}~~\textbf{Exercice #1}, colframe=YellowGreen, colback=white, colbacktitle=YellowGreen!15!white, coltitle=black, boxrule=0.1mm, titlerule=0mm]}{\end{tcolorbox}}

\newenvironment{attention}[1]{\begin{tcolorbox}[title= {\color{Dandelion} \faExclamationTriangle}~~\textbf{Attention #1}, colframe=Dandelion, colback=white, colbacktitle=Dandelion!15!white, coltitle=black, boxrule=0.1mm, titlerule=0mm]}{\end{tcolorbox}}

\newenvironment{avertissement}[1]{\begin{tcolorbox}[title= {\color{RedOrange} \faSkullCrossbones}~~\textbf{#1}, colframe=RedOrange, colback=white, colbacktitle=RedOrange!15!white, coltitle=black, boxrule=0.1mm, titlerule=0mm]}{\end{tcolorbox}}

\newenvironment{theoreme}[1]{\begin{tcolorbox}[title= {\color{red} \faBolt}~~\textbf{Thèorème #1}, colframe=red, colback=white, colbacktitle=red!10!white, coltitle=black, boxrule=0.1mm, titlerule=0mm]}{\end{tcolorbox}}

\newenvironment{exemple}[1]{\begin{tcolorbox}[title= {\color{Periwinkle} \faFlask}~~\textbf{Exemple #1}, colframe=Periwinkle, colback=white, colbacktitle=Periwinkle!10!white, coltitle=black, boxrule=0.1mm, titlerule=0mm]}{\end{tcolorbox}}

\newenvironment{citations}[1]{\begin{tcolorbox}[title= {\color{Gray} \faQuoteRight}~~\textbf{Citation #1}, colframe=Gray, colback=white, colbacktitle=Gray!10!white, coltitle=black, boxrule=0.1mm, titlerule=0mm]}{\end{tcolorbox}}

\geometry{ tmargin=2cm,bmargin=2cm, hmargin=1.5cm }

\everymath{\displaystyle}

\setlength{\parindent}{0cm}

\pagestyle{fancy}

\begin{document}
    
    
    \lhead{Première NSI} \rhead{2022/2023}
    \chead{} 
    \cfoot{}
    \lfoot{Lycée \'Emile Duclaux}
    \rfoot{Page \thepage/\pageref{LastPage}}
    \renewcommand{\headrulewidth}{0pt}
    \renewcommand{\footrulewidth}{0pt}
    
    \Huge S2 - Ch 4. : Booléens  

    \vspace{.25cm}
    \normalsize Execices  
    
    \vspace{.25cm}
    \hrule
    
    \vspace{.5cm}

\emph{Les exercices précédés du symbole \faDesktop
sont à faire sur machine, en sauvegardant le fichier si nécessaire.}

\emph{Les exercices précédés du symbole \faPencil* doivent
être résolus par écrit.}

\hypertarget{octicons-pencil-16-exercice-1}{%
\subsection*{\faPencil* Exercice
1}\label{octicons-pencil-16-exercice-1}}

\begin{enumerate}
\def\labelenumi{\arabic{enumi}.}
\tightlist
\item
  Construire la table de vérité de l'expression : a OU (NON b)
\item
  Construire la table de vérité de l'expression : NON a ET (b OU c)
\item
  Construire la table de vérité de l'expression : (a ET NON b) OU (NON a
  ET b)
\item
  Construire la table de vérité de l'expression : (a OU b) ET (a OU c)
\end{enumerate}

 

\hypertarget{octicons-pencil-16-exercice-2}{%
\subsection*{\faPencil* Exercice
2}\label{octicons-pencil-16-exercice-2}}

Donner la valeur des expressions booléennes suivantes :

\begin{center}
\begin{minipage}{6cm}
\begin{lstlisting}
>>> (1 > 2) and (3 < 5)
>>> ((4 - 7) >= 2) or (2 != 1 + 1)
>>> a = 223
>>> b = 455
>>> a != (b // 2)
\end{lstlisting}
\end{minipage}
\end{center}

\hypertarget{octicons-pencil-16-exercice-3}{%
\subsection*{\faPencil* Exercice
3}\label{octicons-pencil-16-exercice-3}}

On considère la table de vérité de l'expression booléenne Z ci-dessous :

\begin{longtable}[]{@{}cc@{}}
\toprule
x & Z(x)\tabularnewline
\midrule
\endhead
0 & 0\tabularnewline
1 & 0\tabularnewline
\bottomrule
\end{longtable}

Exprimer Z à l'aide des fonctions booléennes ET, OU, NON.

 

\hypertarget{octicons-pencil-16-exercice-4}{%
\subsection*{\faPencil* Exercice
4}\label{octicons-pencil-16-exercice-4}}

On considère la table de vérité de l'expression U ci-dessous :

\begin{longtable}[]{@{}cc@{}}
\toprule
x & U(x)\tabularnewline
\midrule
\endhead
0 & 1\tabularnewline
1 & 1\tabularnewline
\bottomrule
\end{longtable}

Exprimer U à l'aide des fonctions booléennes ET, OU, NON.

\newpage
 

\hypertarget{octicons-pencil-16-exercice-5}{%
\subsection*{\faPencil* Exercice
5}\label{octicons-pencil-16-exercice-5}}

On considère l'extrait de code suivant :

\begin{center}
\begin{minipage}{5cm}
  

\begin{lstlisting}
while (a < 20) or (b > 50):
    ......
    ......
\end{lstlisting}
\end{minipage}
\end{center}
Quelles conditions permettent de mettre fin à cette boucle ?

\begin{itemize}
\tightlist
\item[$\square$]
  la boucle prend fin lorsque \(a < 20\) ou \(b > 50\)
\item[$\square$]
  la boucle prend fin lorsque \(a < 20\) et \(b > 50\)
\item[$\square$]
  la boucle prend fin lorsque \(a \geqslant 20\) ou \(b \leqslant 50\)
\item[$\square$]
  la boucle prend fin lorsque \(a \geqslant 20\) et \(b \leqslant 50\)
\end{itemize}

 

\hypertarget{octicons-pencil-16-exercice-6}{%
\subsection*{\faPencil* Exercice
6}\label{octicons-pencil-16-exercice-6}}

Si A et B sont des variables booléennes, laquelle de ces expressions
booléennes est équivalente à (not A) or B ?

\begin{itemize}
\tightlist
\item[$\square$]
  (A and B) or (not A and B)
\item[$\square$]
  (A and B) or (not A and B) or (not A and not B)
\item[$\square$]
  (not A and B) or (not A and not B)
\item[$\square$]
  (A and B) or (not A and not B)
\end{itemize}

 

\hypertarget{octicons-pencil-16-exercice-7}{%
\subsection*{\faPencil* Exercice
7}\label{octicons-pencil-16-exercice-7}}

Quelle table de vérité correspond à l'expression (NON(A) OU B) ?

\emph{Remarque : dans les tables proposées, la première colonne donne
les valeurs de A, la première ligne les valeurs de B.}

\begin{itemize}
\item[$\square$]
  Table 1 :

  \begin{longtable}[]{@{}lll@{}}
  \toprule
  A/B & 0 & 1\tabularnewline
  \midrule
  \endhead
  0 & 0 & 1\tabularnewline
  1 & 1 & 1\tabularnewline
  \bottomrule
  \end{longtable}
\item[$\square$]
  Table 2 :

  \begin{longtable}[]{@{}lll@{}}
  \toprule
  A/B & 0 & 1\tabularnewline
  \midrule
  \endhead
  0 & 1 & 1\tabularnewline
  1 & 0 & 0\tabularnewline
  \bottomrule
  \end{longtable}
\item[$\square$]
  Table 3 :

  \begin{longtable}[]{@{}lll@{}}
  \toprule
  A/B & 0 & 1\tabularnewline
  \midrule
  \endhead
  0 & 1 & 1\tabularnewline
  1 & 0 & 1\tabularnewline
  \bottomrule
  \end{longtable}
\item[$\square$]
  Table 4 :

  \begin{longtable}[]{@{}lll@{}}
  \toprule
  A/B & 0 & 1\tabularnewline
  \midrule
  \endhead
  0 & 1 & 0\tabularnewline
  1 & 1 & 0\tabularnewline
  \bottomrule
  \end{longtable}
\end{itemize}

 

\hypertarget{octicons-pencil-16-exercice-8}{%
\subsection*{\faPencil* Exercice
8}\label{octicons-pencil-16-exercice-8}}

Parmi les quatre expressions suivantes, laquelle s'évalue en
\lstinline!True! ?

\begin{itemize}
\tightlist
\item[$\square$]
  \lstinline!False and (True and False)!
\item[$\square$]
  \lstinline!False or (True and False)!
\item[$\square$]
  \lstinline!True and (True and False)!
\item[$\square$]
  \lstinline!True or (True and False)!
\end{itemize}

 

\hypertarget{octicons-pencil-16-exercice-9}{%
\subsection*{\faPencil* Exercice
9}\label{octicons-pencil-16-exercice-9}}

À quelle affectation sont équivalentes les instructions suivantes, où
\texttt{a}, \texttt{b} sont des variables entières et \texttt{c} une
variable booléenne ?

\begin{center}
\begin{minipage}{3cm}
\begin{lstlisting}
if a==b:
    c = True
elif a > b+10:
    c = True
else:
    c = False
\end{lstlisting}
\end{minipage}
\end{center}

\begin{itemize}
\tightlist
\item[$\square$]
  \lstinline!c = (a==b) or (a > b+10)!
\item[$\square$]
  \lstinline!c = (a==b) and (a > b+10)!
\item[$\square$]
  \lstinline!c = not(a==b)!
\item[$\square$]
  \lstinline!c = not(a > b+10)!
\end{itemize}



\end{document}