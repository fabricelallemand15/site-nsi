\documentclass[11pt,a4paper]{article}

\usepackage{etex} %évite les erreurs no room for ....

\usepackage[french]{babel}
\usepackage[utf8]{inputenc}
\usepackage[T1]{fontenc}
\usepackage{hyperref}
\usepackage{amsmath}
\usepackage{amssymb} %gestion des symboles mathmatiques
\usepackage{amsfonts} %gestion des polices mathmatiques
\usepackage{fancybox} %gestion des encadrements
\usepackage{lastpage} %gestion du nombre total de pages du document
\usepackage{geometry} %gestion des marges du document
\usepackage{fancyhdr} %gestion des entetes et des pieds de page
\usepackage{epsfig}
\usepackage[dvipsnames]{xcolor}
\usepackage{pdftricks}
\usepackage{eurosym}
\usepackage{pifont}
\usepackage{eucal}
\usepackage{multirow}
\usepackage{tabularx}
\usepackage{ulem}
\usepackage{dcolumn}
\usepackage{textcomp}
\usepackage{diagbox}
\usepackage{lscape}
\usepackage{ifpdf}
\usepackage[tikz]{bclogo}
\usepackage{amsthm}
\usepackage{tabvar}
\usepackage{multicol}
\usepackage{frcursive}
\usepackage{color}
\usepackage{colortbl}
\usepackage{pgf,tikz}
\usetikzlibrary{arrows}
\usepackage{eurosym}
\usepackage{cancel}
\usepackage{longtable,booktabs}
\usepackage{tcolorbox}
\tcbuselibrary{skins}
\tcbuselibrary{breakable}
\usepackage{varwidth}
\usepackage{MnSymbol} % contient le caractère \nsubset
\usepackage{wrapfig}
\usepackage{wasysym} % pour certains symboles
\usepackage{minted}
\usepackage{titlesec}
\usepackage[np]{numprint}
\usepackage{fontawesome5}
\usepackage{xspace}

\titleformat{\section}
{\Large\bfseries}{\thesection}{1em}{}[\titlerule]
% {\Large\bfseries}{\thesection}{1em}{\MakeUppercase}[\titlerule]

\titleformat{\subsection}
{\large\bfseries}{\thesubsection}{1em}{}[\titlerule]
% {\large\bfseries}{\thesubsection}{1em}{\MakeUppercase}[\titlerule]

%% Macro pour limite  gauche%%
\newcommand{\limgauche}[2]{\lim_{#1\rightarrow #2\hspace{-1.5em}\raisebox{-1.2ex}{\scriptsize $#1<#2$}}}

%% Macro pour limite  droite%%
\newcommand{\limdroite}[2]{\lim_{#1\rightarrow #2\hspace{-1.5em}\raisebox{-1.2ex}{\scriptsize $#1>#2$}}}

\newcommand{\R}{\mathbb{R}}
\newcommand{\N}{\mathbb{N}}
%\newcommand{\D}{\mathbb{D}}
\newcommand{\Z}{\mathbb{Z}}
\newcommand{\Q}{\mathbb{Q}}
\newcommand{\C}{\mathbb{C}}
\newcommand{\lm}[2]{\displaystyle{\lim_{#1 \rightarrow #2}}}
\newcommand{\e}[1]{\text{e}^{#1}}
\renewcommand{\theenumi}{\textbf{\arabic{enumi}}}
\renewcommand{\labelenumi}{\textbf{\theenumi.}}
\renewcommand{\theenumii}{\textbf{\alph{enumii}}}
\renewcommand{\labelenumii}{\textbf{\theenumii.}}
\newcommand{\vect}[1]{\mathchoice%
{\overrightarrow{\displaystyle\mathstrut#1\,\,}}%
{\overrightarrow{\textstyle\mathstrut#1\,\,}}%
{\overrightarrow{\scriptstyle\mathstrut#1\,\,}}%
{\overrightarrow{\scriptscriptstyle\mathstrut#1\,\,}}}
\def\Oij{$\left(\text{O},~\vect{\imath},~\vect{\jmath}\right)$}
\def\Oijk{$\left(\text{O},~\vect{\imath},~ \vect{\jmath},~ \vect{k}\right)$}
\def\Ouv{$\left(\text{O},~\vect{u},~\vect{v}\right)$}
\newcommand{\V}{\overrightarrow}
\newcommand{\Coor}{\binom}
\newcommand{\Rep}{\left(O;\V{i},\V{j}\right)}
\renewcommand{\emph}{\textit}
\setlength{\emergencystretch}{3em} % prevent overfull lines
\providecommand{\tightlist}{\setlength{\itemsep}{0pt}\setlength{\parskip}{0pt}}

\newcommand{\Syst}[2]{\left\{
    \begin{array}{ccccc}
        #1   
        #2
    \end{array}\right.
}

%%%%%%%%%%%%%%%%%% MES ENVIRONNEMENTS %%%%%%%%%%%%%%%%%%%%%%%%%%%%%

\newenvironment{definition}[1]{\begin{tcolorbox}[title= {\color{NavyBlue} \faPencil*}~~\textbf{Définition #1}, colframe=NavyBlue, colback=white, colbacktitle=NavyBlue!10!white, coltitle=black, boxrule=0.1mm, titlerule=0mm]}{\end{tcolorbox}}

\newenvironment{info}[1]{\begin{tcolorbox}[title= {\color{ProcessBlue} \faBook}~~\textbf{#1}, colframe=ProcessBlue, colback=white, colbacktitle=ProcessBlue!15!white, coltitle=black, boxrule=0.1mm, titlerule=0mm]}{\end{tcolorbox}}

\newenvironment{remarque}[1]{\begin{tcolorbox}[title= {\color{SkyBlue} \faInfo}~~\textbf{Remarque #1}, colframe=SkyBlue, colback=white, colbacktitle=SkyBlue!20!white, coltitle=black, boxrule=0.1mm, titlerule=0mm]}{\end{tcolorbox}}

\newenvironment{indication}[1]{\begin{tcolorbox}[title= {\color{Turquoise} \faHotjar}~~\textbf{Indication #1}, colframe=Turquoise, colback=white, colbacktitle=Turquoise!15!white, coltitle=black, boxrule=0.1mm, titlerule=0mm]}{\end{tcolorbox}}

\newenvironment{solution}[1]{\begin{tcolorbox}[title= {\color{ForestGreen} \faCheck}~~\textbf{Solution #1}, colframe=ForestGreen, colback=white, colbacktitle=ForestGreen!15!white, coltitle=black, boxrule=0.1mm, titlerule=0mm]}{\end{tcolorbox}}

\newenvironment{exercice}[1]{\begin{tcolorbox}[title= {\color{YellowGreen} \faQuestion}~~\textbf{Exercice #1}, colframe=YellowGreen, colback=white, colbacktitle=YellowGreen!15!white, coltitle=black, boxrule=0.1mm, titlerule=0mm]}{\end{tcolorbox}}

\newenvironment{attention}[1]{\begin{tcolorbox}[title= {\color{Dandelion} \faExclamationTriangle}~~\textbf{Attention #1}, colframe=Dandelion, colback=white, colbacktitle=Dandelion!15!white, coltitle=black, boxrule=0.1mm, titlerule=0mm]}{\end{tcolorbox}}

\newenvironment{avertissement}[1]{\begin{tcolorbox}[title= {\color{RedOrange} \faSkullCrossbones}~~\textbf{#1}, colframe=RedOrange, colback=white, colbacktitle=RedOrange!15!white, coltitle=black, boxrule=0.1mm, titlerule=0mm]}{\end{tcolorbox}}

\newenvironment{theoreme}[1]{\begin{tcolorbox}[title= {\color{red} \faBolt}~~\textbf{Thèorème #1}, colframe=red, colback=white, colbacktitle=red!10!white, coltitle=black, boxrule=0.1mm, titlerule=0mm]}{\end{tcolorbox}}

\newenvironment{exemple}[1]{\begin{tcolorbox}[title= {\color{Periwinkle} \faFlask}~~\textbf{Exemple #1}, colframe=Periwinkle, colback=white, colbacktitle=Periwinkle!10!white, coltitle=black, boxrule=0.1mm, titlerule=0mm]}{\end{tcolorbox}}

\newenvironment{citations}[1]{\begin{tcolorbox}[title= {\color{Gray} \faQuoteRight}~~\textbf{Citation #1}, colframe=Gray, colback=white, colbacktitle=Gray!10!white, coltitle=black, boxrule=0.1mm, titlerule=0mm]}{\end{tcolorbox}}

\geometry{ tmargin=2cm,bmargin=2cm, hmargin=1.5cm }

\everymath{\displaystyle}

\setlength{\parindent}{0cm}

\pagestyle{fancy}

\begin{document}
    
    
    \lhead{Première NSI} \rhead{2022/2023}
    \chead{} 
    \cfoot{}
    \lfoot{Lycée \'Emile Duclaux}
    \rfoot{Page \thepage/\pageref{LastPage}}
    \renewcommand{\headrulewidth}{0pt}
    \renewcommand{\footrulewidth}{0pt}

\Huge S2 - Ch. 4 : Booléens

\vspace{.25cm}
\normalsize Cours

\vspace{.25cm}
\hrule

\vspace{.5cm}


Nous avons déjà rencontré les booléens dans la première séquence.

En programmation informatique, un booléen est un type de variable à deux
états (généralement notés vrai et faux), destiné à représenter les
valeurs de vérité de la logique et l'algèbre booléenne. Il est nommé
ainsi d'après George Boole (1815-1864), fondateur dans le milieu du XIXe
siècle de l'algèbre portant son nom.

Nous avons vu qu'en Python, les deux valeurs booléennes sont notées
\texttt{True} et \texttt{False}.

De manière équivalents, on adopte souvent une notation numérique en
associant 1 à \texttt{True} et 0 à \texttt{False}.

\hypertarget{opuxe9rateurs-booluxe9ens-de-base}{%
\subsection*{1. Opérateurs booléens de
base}\label{opuxe9rateurs-booluxe9ens-de-base}}

Dans le cours sur les bases de Python, nous avons déjà vu les opérateurs
\texttt{or}, \texttt{and} et \texttt{not}.

\hypertarget{opuxe9rateur-ou}{%
\subsubsection*{Opérateur OU}\label{opuxe9rateur-ou}}

\begin{definition}{}

Soit \(a\) et \(b\) deux expressions :


$$a\textrm{ OU }b\textrm{ est vrai }\iff a\textrm{ est vrai ou }b\textrm{ est vrai}$$
\end{definition}

Table de vérité de l'opérateur OU :

\begin{longtable}[]{@{}ccc@{}}
\toprule
\(a\) & \(b\) & \(a\) OU \(b\)\tabularnewline
\midrule
\endhead
1 & 1 & 1\tabularnewline
1 & 0 & 1\tabularnewline
0 & 1 & 1\tabularnewline
0 & 0 & 0\tabularnewline
\bottomrule
\end{longtable}

\begin{remarque}{}
En logique l'opérateur OU est \textbf{inclusif}
: cela signifie que \(a\) OU \(b\) est vrai aussi lorsque \(a\) est vrai
et \(b\) est vrai. Dans la langue courant, le mot \emph{ou} est le plus
souvent \textbf{exclusif} : dans un menu, par exemple ``fromage ou
dessert'' ne permet pas de prendre les deux.

\end{remarque}

\hypertarget{opuxe9rateur-et}{%
\subsubsection*{Opérateur ET}\label{opuxe9rateur-et}}

\begin{definition}{}
    Soit \(a\) et \(b\) deux expressions :

 
$$a\textrm{ ET }b\textrm{ est vrai }\iff a\textrm{ est vrai et }b\textrm{ est vrai}$$
 
\end{definition}


Table de vérité de l'opérateur ET :

\begin{longtable}[]{@{}ccc@{}}
\toprule
\(a\) & \(b\) & \(a\) ET \(b\)\tabularnewline
\midrule
\endhead
1 & 1 & 1\tabularnewline
1 & 0 & 0\tabularnewline
0 & 1 & 0\tabularnewline
0 & 0 & 0\tabularnewline
\bottomrule
\end{longtable}

\hypertarget{opuxe9rateur-non}{%
\subsubsection*{Opérateur NON}\label{opuxe9rateur-non}}


\begin{definition}{} 
    Soit \(a\) une expression :

 
$$(\textrm{NON }a)\textrm{ est vrai }\iff a\textrm{ est faux}$$
\end{definition}

Table de vérité de l'opérateur NON :

\begin{longtable}[]{@{}cc@{}}
\toprule
\(a\) & NON \(a\)\tabularnewline
\midrule
\endhead
1 & 0\tabularnewline
0 & 1\tabularnewline
\bottomrule
\end{longtable}

\hypertarget{expressions-booluxe9ennes}{%
\subsection*{2. Expressions booléennes}\label{expressions-booluxe9ennes}}

Les opérateurs de base peuvent être combinés pour formuler des
expressions booléennes plus complexes. Pour éviter des problèmes
d'interprétation, il est préférable d'utiliser des parenthèses pour
marquer les priorités.

\begin{exercice}{}
 
Recopier et compléter la table de vérité ci-dessous :

\begin{longtable}[]{@{}ccccccc@{}}
    \toprule
    \(a\) & \(b\) & NON \(a\) & NON \(b\) & (NON \(a\)) ET (NON \(b\)) &
    NON((NON \(a\)) ET (NON \(b\))) & \(a\) OU \(b\)\tabularnewline
    \midrule
    \endhead
    1 & 1 & & & & &\tabularnewline
    1 & 0 & & & & &\tabularnewline
    0 & 1 & & & & &\tabularnewline
    0 & 0 & & & & &\tabularnewline
    \bottomrule
    \end{longtable}

Que peut-on constater ?
\end{exercice}

\hypertarget{le-ou-exclusif}{%
\subsection*{3. Le ou exclusif}\label{le-ou-exclusif}}

Le OU logique étant inclusif, on définit un opérateur spécifique pour le
ou exclusif, appelé opérateur XOR.

\begin{definition}{} Soit \(a\) et \(b\) deux expressions :

 
$$a\textrm{ XOR }b\textrm{ est vrai }\iff (a\textrm{ est vrai et }b\textrm{ est faux})\textrm{ ou }(a\textrm{ est faux et }b\textrm{ est vrai})$$
\end{definition}

Table de vérité de l'opérateur XOR :

\begin{longtable}[]{@{}ccc@{}}
\toprule
\(a\) & \(b\) & \(a\) XOR \(b\)\tabularnewline
\midrule
\endhead
1 & 1 & 0\tabularnewline
1 & 0 & 1\tabularnewline
0 & 1 & 1\tabularnewline
0 & 0 & 0\tabularnewline
\bottomrule
\end{longtable}

En Python, l'opérateur \texttt{xor} n'existe pas. Le ou exclusif est
noté \texttt{\^{}}.

\begin{minted}{pycon} 
>>> True ^ False
True
\end{minted}
 

\hypertarget{laddition-binaire-en-mode-booluxe9en}{%
\subsection*{4. L'addition binaire en mode
booléen}\label{laddition-binaire-en-mode-booluxe9en}}

Lorsque nous posons l'addition de deux entiers écrits en base 2, nous
avons besoin d'additionner des groupes de 3 bits (un pour chaque nombre
et un pour la retenue).

Voyons ce que donne l'addition de trois bits :

\begin{longtable}[]{@{}cccc@{}}
\toprule
\(a\) & \(b\) & \(c\) & \(a+b+c\)\tabularnewline
\midrule
\endhead
0 & 0 & 0 & 0\tabularnewline
1 & 0 & 0 & 1\tabularnewline
0 & 1 & 0 & 1\tabularnewline
0 & 0 & 1 & 1\tabularnewline
1 & 1 & 0 & 10\tabularnewline
1 & 0 & 1 & 10\tabularnewline
0 & 1 & 1 & 10\tabularnewline
1 & 1 & 1 & 11\tabularnewline
\bottomrule
\end{longtable}

\begin{exercice}{}
En assimilant 0 à \texttt{False} et 1 à
\texttt{True} écrire une fonction \texttt{add\_trois\_bits(a,\ b,\ c)}
qui retourne la somme \(a+b+c\) en utilisant uniquement les opérateurs
ET, OU et NON. On retournera la somme sous la forme d'une chaîne de deux
caractères (\texttt{"01"} par exemple).

\begin{minted}{python}
def add_3_bits(a, b, c) :
    unite = int(...)
    deuzaine = int(...)
    return str(deuzaine)+str(unite)

assert add_3_bits(0,0,0)=="00"
assert add_3_bits(1,0,0)=="01"
assert add_3_bits(0,1,0)=="01"
assert add_3_bits(0,0,1)=="01"
assert add_3_bits(1,1,0)=="10"
assert add_3_bits(1,0,1)=="10"
assert add_3_bits(0,1,1)=="10"
assert add_3_bits(1,1,1)=="11"
print("C'est parfait !")
\end{minted}
 
\end{exercice}

\end{document}