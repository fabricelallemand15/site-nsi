\documentclass[11pt,a4paper]{article}

\usepackage{etex} %évite les erreurs no room for ....

\usepackage[french]{babel}
\usepackage[utf8]{inputenc}
% \usepackage[T1]{fontenc}
\usepackage{hyperref}
\usepackage{amsmath}
\usepackage{amssymb} %gestion des symboles mathmatiques
\usepackage{amsfonts} %gestion des polices mathmatiques
\usepackage{fancybox} %gestion des encadrements
\usepackage{lastpage} %gestion du nombre total de pages du document
\usepackage{geometry} %gestion des marges du document
\usepackage{fancyhdr} %gestion des entetes et des pieds de page
\usepackage{epsfig}
\usepackage[dvipsnames]{xcolor}
\usepackage{pdftricks}
\usepackage{eurosym}
\usepackage{pifont}
\usepackage{eucal}
\usepackage{multirow}
\usepackage{tabularx}
\usepackage{ulem}
\usepackage{dcolumn}
\usepackage{textcomp}
\usepackage{diagbox}
\usepackage{lscape}
\usepackage{ifpdf}
\usepackage[tikz]{bclogo}
\usepackage{amsthm}
\usepackage{tabvar}
\usepackage{multicol}
\usepackage{frcursive}
\usepackage{color}
\usepackage{colortbl}
\usepackage{pgf,tikz}
\usetikzlibrary{arrows}
\usepackage{eurosym}
\usepackage{cancel}
\usepackage{longtable,booktabs}
\usepackage{tcolorbox}
\tcbuselibrary{skins}
\tcbuselibrary{breakable}
\usepackage{varwidth}
\usepackage{MnSymbol} % contient le caractère \nsubset
\usepackage{wrapfig}
\usepackage{wasysym} % pour certains symboles
\usepackage{minted}
\usepackage{titlesec}
\usepackage[np]{numprint}
\usepackage{fontawesome5}
\usepackage{xspace}
\usepackage{listingsutf8}

\titleformat{\section}
{\Large\bfseries}{\thesection}{1em}{}[\titlerule]
% {\Large\bfseries}{\thesection}{1em}{\MakeUppercase}[\titlerule]

\titleformat{\subsection}
{\large\bfseries}{\thesubsection}{1em}{}[\titlerule]
% {\large\bfseries}{\thesubsection}{1em}{\MakeUppercase}[\titlerule]

%% Macro pour limite  gauche%%
\newcommand{\limgauche}[2]{\lim_{#1\rightarrow #2\hspace{-1.5em}\raisebox{-1.2ex}{\scriptsize $#1<#2$}}}

%% Macro pour limite  droite%%
\newcommand{\limdroite}[2]{\lim_{#1\rightarrow #2\hspace{-1.5em}\raisebox{-1.2ex}{\scriptsize $#1>#2$}}}

\newcommand{\R}{\mathbb{R}}
\newcommand{\N}{\mathbb{N}}
%\newcommand{\D}{\mathbb{D}}
\newcommand{\Z}{\mathbb{Z}}
\newcommand{\Q}{\mathbb{Q}}
\newcommand{\C}{\mathbb{C}}
\newcommand{\lm}[2]{\displaystyle{\lim_{#1 \rightarrow #2}}}
\newcommand{\e}[1]{\text{e}^{#1}}
\renewcommand{\theenumi}{\textbf{\arabic{enumi}}}
\renewcommand{\labelenumi}{\textbf{\theenumi.}}
\renewcommand{\theenumii}{\textbf{\alph{enumii}}}
\renewcommand{\labelenumii}{\textbf{\theenumii.}}
\newcommand{\vect}[1]{\mathchoice%
{\overrightarrow{\displaystyle\mathstrut#1\,\,}}%
{\overrightarrow{\textstyle\mathstrut#1\,\,}}%
{\overrightarrow{\scriptstyle\mathstrut#1\,\,}}%
{\overrightarrow{\scriptscriptstyle\mathstrut#1\,\,}}}
\def\Oij{$\left(\text{O},~\vect{\imath},~\vect{\jmath}\right)$}
\def\Oijk{$\left(\text{O},~\vect{\imath},~ \vect{\jmath},~ \vect{k}\right)$}
\def\Ouv{$\left(\text{O},~\vect{u},~\vect{v}\right)$}
\newcommand{\V}{\overrightarrow}
\newcommand{\Coor}{\binom}
\newcommand{\Rep}{\left(O;\V{i},\V{j}\right)}
\renewcommand{\emph}{\textit}
\setlength{\emergencystretch}{3em} % prevent overfull lines
\providecommand{\tightlist}{\setlength{\itemsep}{0pt}\setlength{\parskip}{0pt}}

\newcommand{\Syst}[2]{\left\{
  \begin{array}{ccccc}
    #1   
    #2
  \end{array}\right.
}

%configuration de listings
\definecolor{couleuroperations}{rgb}{0.6,0.2,1}
\lstset{
backgroundcolor=\color{lightgray!15},
frame=leftline,
%   framexleftmargin=5mm,
rulesepcolor=\color{lightgray},
inputencoding=utf8,
language=python,
basicstyle=\ttfamily\small, %
identifierstyle=\color{black}, %
keywordstyle=\color{MidnightBlue}, %
stringstyle=\color{purple}, %
commentstyle=\it\color{olive}, %
columns=flexible, %
tabsize=2, %
extendedchars=true, %
showspaces=false, %
showstringspaces=false, %
numbers=none, %
numberstyle=\tiny, %
breaklines=true, %
breakautoindent=true, %
captionpos=b,
literate=%
  *{+}{{{\color{couleuroperations}+}}}1
  {-}{{{\color{couleuroperations}-}}}1
  {/}{{{\color{couleuroperations}/}}}1
  {*}{{{\color{couleuroperations}*}}}1
  {=}{{{\color{couleuroperations}=}}}1
  {<}{{{\color{couleuroperations}<}}}1
  {>}{{{\color{couleuroperations}>}}}1
  {\%}{{{\color{couleuroperations}\%}}}1
  {(}{{{\color{couleuroperations}(}}}1
  {)}{{{\color{couleuroperations})}}}1
  {[}{{{\color{couleuroperations}[}}}1
  {]}{{{\color{couleuroperations}]}}}1
  {.}{{{\color{couleuroperations}.}}}1
  {,}{{{\color{couleuroperations},}}}1
  {1}{{{\color{Cyan}1}}}1
  {2}{{{\color{Cyan}2}}}1
  {3}{{{\color{Cyan}3}}}1
  {4}{{{\color{Cyan}4}}}1
  {5}{{{\color{Cyan}5}}}1
  {6}{{{\color{Cyan}6}}}1
  {7}{{{\color{Cyan}7}}}1
  {8}{{{\color{Cyan}8}}}1
  {9}{{{\color{Cyan}9}}}1
  {0}{{{\color{Cyan}0}}}1,
}

%%%%%%%%%%%%%%%%%% MES ENVIRONNEMENTS %%%%%%%%%%%%%%%%%%%%%%%%%%%%%

\newenvironment{definition}[1]{\begin{tcolorbox}[title= {\color{NavyBlue} \faPencil*}~~\textbf{Définition #1}, colframe=NavyBlue, colback=white, colbacktitle=NavyBlue!10!white, coltitle=black, boxrule=0.1mm, titlerule=0mm]}{\end{tcolorbox}}

\newenvironment{info}[1]{\begin{tcolorbox}[title= {\color{ProcessBlue} \faBook}~~\textbf{#1}, colframe=ProcessBlue, colback=white, colbacktitle=ProcessBlue!15!white, coltitle=black, boxrule=0.1mm, titlerule=0mm]}{\end{tcolorbox}}

\newenvironment{remarque}[1]{\begin{tcolorbox}[title= {\color{SkyBlue} \faInfo}~~\textbf{Remarque #1}, colframe=SkyBlue, colback=white, colbacktitle=SkyBlue!20!white, coltitle=black, boxrule=0.1mm, titlerule=0mm]}{\end{tcolorbox}}

\newenvironment{indication}[1]{\begin{tcolorbox}[title= {\color{Turquoise} \faHotjar}~~\textbf{Indication #1}, colframe=Turquoise, colback=white, colbacktitle=Turquoise!15!white, coltitle=black, boxrule=0.1mm, titlerule=0mm]}{\end{tcolorbox}}

\newenvironment{solution}[1]{\begin{tcolorbox}[title= {\color{ForestGreen} \faCheck}~~\textbf{Solution #1}, colframe=ForestGreen, colback=white, colbacktitle=ForestGreen!15!white, coltitle=black, boxrule=0.1mm, titlerule=0mm]}{\end{tcolorbox}}

\newenvironment{exercice}[1]{\begin{tcolorbox}[title= {\color{YellowGreen} \faQuestion}~~\textbf{Exercice #1}, colframe=YellowGreen, colback=white, colbacktitle=YellowGreen!15!white, coltitle=black, boxrule=0.1mm, titlerule=0mm]}{\end{tcolorbox}}

\newenvironment{attention}[1]{\begin{tcolorbox}[title= {\color{Dandelion} \faExclamationTriangle}~~\textbf{Attention #1}, colframe=Dandelion, colback=white, colbacktitle=Dandelion!15!white, coltitle=black, boxrule=0.1mm, titlerule=0mm]}{\end{tcolorbox}}

\newenvironment{avertissement}[1]{\begin{tcolorbox}[title= {\color{RedOrange} \faSkullCrossbones}~~\textbf{#1}, colframe=RedOrange, colback=white, colbacktitle=RedOrange!15!white, coltitle=black, boxrule=0.1mm, titlerule=0mm]}{\end{tcolorbox}}

\newenvironment{theoreme}[1]{\begin{tcolorbox}[title= {\color{red} \faBolt}~~\textbf{Thèorème #1}, colframe=red, colback=white, colbacktitle=red!10!white, coltitle=black, boxrule=0.1mm, titlerule=0mm]}{\end{tcolorbox}}

\newenvironment{exemple}[1]{\begin{tcolorbox}[title= {\color{Periwinkle} \faFlask}~~\textbf{Exemple #1}, colframe=Periwinkle, colback=white, colbacktitle=Periwinkle!10!white, coltitle=black, boxrule=0.1mm, titlerule=0mm]}{\end{tcolorbox}}

\newenvironment{citations}[1]{\begin{tcolorbox}[title= {\color{Gray} \faQuoteRight}~~\textbf{Citation #1}, colframe=Gray, colback=white, colbacktitle=Gray!10!white, coltitle=black, boxrule=0.1mm, titlerule=0mm]}{\end{tcolorbox}}

\geometry{ tmargin=2cm,bmargin=2cm, hmargin=1.5cm }

\everymath{\displaystyle}

\setlength{\parindent}{0cm}

\pagestyle{fancy}

\begin{document}
    
    
    \lhead{Première NSI} \rhead{2022/2023}
    \chead{} 
    \cfoot{}
    \lfoot{Lycée \'Emile Duclaux}
    \rfoot{Page \thepage/\pageref{LastPage}}
    \renewcommand{\headrulewidth}{0pt}
    \renewcommand{\footrulewidth}{0pt}
    
    \Huge S2 - Ch. 5 : Représentation d'un texte en machine  

    \vspace{.25cm}
    \normalsize Exercices  
    
    \vspace{.25cm}
    \hrule
    
    \vspace{.5cm}

  


\emph{Les exercices précédés du symbole \faDesktop{}
sont à faire sur machine, en sauvegardant le fichier si nécessaire.}

\emph{Les exercices précédés du symbole \faPencil* doivent
être résolus par écrit.}

\hypertarget{fontawesome-solid-computer-exercice-1}{%
\subsection*{\faDesktop{} Exercice
1}\label{fontawesome-solid-computer-exercice-1}}

Écrire une procédure qui affiche à l'écran la table des 128 caractères
ASCII sur 8 lignes de 16 colonnes, avec deux boucles \texttt{for}
imbriquées et la fonction \texttt{chr()}.

 

\hypertarget{fontawesome-solid-computer-exercice-2}{%
\subsection*{\faDesktop{} Exercice
2}\label{fontawesome-solid-computer-exercice-2}}

L'encodage consiste à coder un caractère par un nombre. Il ne faut pas
confondre cette opération de \textbf{numérisation} avec le
\textbf{chiffrement}, ou \textbf{cryptage}, qui consiste aussi à
remplacer des caractères par des nombres, mais de telle façon que le
décodage ne puisse se faire que par le destinataire du message. La
numérisation est cependant le plus souvent la première étape du
chiffrement.

Considérons le procédé \textbf{rot-13} qui consiste à remplacer une
lettre codée par l'entier \(n\) par la lettre codée par l'entier
\(n+13\) (on se limite aux caractères de la table ASCII).

Comme l'alphabet latin compte 26 lettres, le déchiffrement se fait
exactement par la même opération.

Pour trouver que l'image par rot-13 du caractère `R' est `E' :

\begin{itemize}
\tightlist
\item
  on calcule d'abord le rang alphabétique du caractère :

\begin{center}
\begin{minipage}{5cm}
\begin{minted}[frame=lines,framesep=2mm, framerule=2pt,rulecolor=Gray!90,bgcolor=Gray!15,label=\faPython{} Console Python]{pycon}
>>> ord('R') - ord('A')
17    
\end{minted}
\end{minipage}
\end{center}
 
\item
  puis on ajoute 13 à ce rang et on prend le reste dans la division
  euclidienne par 26 :

\begin{center}
\begin{minipage}{8cm}
\begin{minted}[mathescape,frame=lines,framesep=2mm, framerule=2pt,rulecolor=Gray!90,bgcolor=Gray!15,label=\faPython{} Console Python]{pycon}
>>> (ord('R') - ord('A') + 13) % 26
4
\end{minted}
\end{minipage}
\end{center}

\item
  enfin on retrouve le rang du caractère associé au rang alphabétique
  calculé :

\begin{center}
\begin{minipage}{5cm}
\begin{minted}[frame=lines,framesep=2mm, framerule=2pt,rulecolor=Gray!90,bgcolor=Gray!15,label=\faPython{} Console Python]{pycon}
>>> chr(ord('A') + 4)
'E'
\end{minted}
\end{minipage}
\end{center}
  \
\end{itemize}

Écrire une fonction \texttt{rot13(chaine)} qui chiffre ou déchiffre la
chaîne (en majuscules ou convertie en majuscules avec
\texttt{chaine.upper())} passée en paramètre, avec l'algorithme rot13.

 

\hypertarget{octicons-pencil-16-exercice-3}{%
\subsection*{\faPencil* Exercice
3}\label{octicons-pencil-16-exercice-3}}

\begin{enumerate}
\def\labelenumi{\arabic{enumi}.}
\tightlist
\item
  Sur une page web qui s'affiche sur notre navigateur on peut lire :

\begin{center}
\begin{minipage}{10cm}
En consÃ\copyright quence, l'AssemblÃ\copyright e Nationale reconnaÃ\copyright t et dÃ\copyright clare, en
prÃ\copyright sence ...
\end{minipage}
\end{center}
 

Quelle peut être la cause des affichages étranges de cette page ?

\begin{itemize}
\tightlist
\item[$\square$]
  l'encodage des caractères n'est pas celui attendu par le navigateur
\item[$\square$]
  le texte original est en japonais
\item[$\square$]
  la taille des caractères n'est pas celle attendue par le navigateur
\item[$\square$]
  la connexion à Internet présente des coupures
\end{itemize}

\newpage

\item
  Le code ASCII permet de représenter en binaire les caractères
  alphanumériques. Quel est son principal inconvénient ?

\begin{itemize}
\tightlist
\item[$\square$]
  Il utilise beaucoup de bits.
\item[$\square$]
  Il ne différencie pas les majuscules des minuscules.
\item[$\square$]
  Il ne représente pas les caractères accentués.
\item[$\square$]
  Il n'est passer compatible avec la plupart des systèmes informatiques
\end{itemize}

\end{enumerate}

\hypertarget{octicons-pencil-16-exercice-4}{%
\subsection*{\faPencil* Exercice
4}\label{octicons-pencil-16-exercice-4}}

\begin{enumerate}
\def\labelenumi{\arabic{enumi}.}
\tightlist
\item
  Parmi les caractères ci-dessous, lequel ne fait pas partie du code
  ASCII ?
\end{enumerate}

\begin{itemize}
\tightlist
\item[$\square$]
  a
\item[$\square$]
  B
\item[$\square$]
  @
\item[$\square$]
  é
\end{itemize}

\begin{enumerate}
\def\labelenumi{\arabic{enumi}.}
\setcounter{enumi}{1}
\tightlist
\item
  Laquelle de ces affirmations concernant le codage UTF-8 des caractères
  est vraie ?
\end{enumerate}

\begin{itemize}
\tightlist
\item[$\square$]
  le codage UTF-8 est sur 7 bits
\item[$\square$]
  le codage UTF-8 est sur 8 bits
\item[$\square$]
  le codage UTF-8 est sur 1 à 4 octets
\item[$\square$]
  le codage UTF-8 est sur 8 octets
\end{itemize}

\begin{enumerate}
\def\labelenumi{\arabic{enumi}.}
\setcounter{enumi}{2}
\tightlist
\item
  Quel est un avantage du codage UTF-8 par rapport au codage ASCII ?
\end{enumerate}

\begin{itemize}
\tightlist
\item[$\square$]
  il permet de coder un caractère sur un octet au lieu de deux
\item[$\square$]
  il permet de coder les majuscules
\item[$\square$]
  il permet de coder tous les caractères
\item[$\square$]
  il permet de coder différentes polices de caractères
\end{itemize}

 

\hypertarget{fontawesome-solid-computer-exercice-5}{%
\subsection*{\faDesktop{} Exercice
5}\label{fontawesome-solid-computer-exercice-5}}

En Python, pour saisir directement un caractère à partir de son point de
code Unicode, on peut utiliser des séquences d'échappement spéciales :
\lstinline!\uxxxx! si le point de code peut s'écrire avec 4 chiffres hexadécimaux,
ou \lstinline!\Uxxxxxxxx! s'il faut plus de quatre chiffres, en remplissant par
des 0 à gauche les positions vides sur les huit possibles.

\begin{enumerate}
\def\labelenumi{\arabic{enumi}.}
\tightlist
\item
  Tester l'instruction ci-dessous dans une console Python :

\begin{center}
\begin{minipage}{5cm}
\begin{minted}[frame=lines,framesep=2mm, framerule=2pt,rulecolor=Gray!90,bgcolor=Gray!15,label=\faPython{} Console Python]{pycon}
>>> print("\U0001f600")
\end{minted}
\end{minipage}
\end{center}
  
\item
  Écrire un code Python qui affiche tous les caractères dont le point de
  code est compris entre \texttt{U+1F600} et \texttt{U+1F64F} sur des
  lignes de 16 caractères par colonne.
\end{enumerate}


\end{document}